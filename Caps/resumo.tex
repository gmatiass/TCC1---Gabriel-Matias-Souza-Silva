%\chapter*{Resumo}

% --- resumo em português ---
\begin{resumo}
A maior parte dos sistemas de visão artificial desenvolvidos nos dias atuais utiliza duas câmeras para o sensoriamento. Esta técnica é conhecida como sistema de visão estereoscópica binocular. Nela são utilizadas duas imagens dispostas em diferentes posições para determinar a estrutura tridimensional dos objetos, utilizando o princípio da triangulação. O princípio da triangulação determina a posição do objeto através da intersecção dos raios correspondentes à projeção na imagem de cada câmera e tem como maior dificuldade identificar os elementos correspondentes nas duas imagens, sendo conhecido como o problema da correspondência estereoscópica. 

Em grande parte, os problemas inerentes de um sistema de visão computacional dependerão de sua aplicação, do tipo de informação relevante da imagem, da metodologia para obter essa informação e quais ações o robô irá tomar a partir disto.

Uma vez que, a informação tridimensional pode ser obtida utilizando apenas uma câmera e um sensor que obtenha uma informação de distância do objeto. Sendo assim, essa técnica proporciona um menor tempo de processamento, o que para um robô móvel é de suma importância. Este projeto de conclusão de curso tem como objetivo implementar um algoritmo de controle em um microcontrolador Raspberry Pi. Este hardware será embarcado em um robô móvel que terá entradas oriundas de sensores ultra-sônicos e  de uma câmera permitindo a identificação do ambiente de maneira autônoma.

\vspace{\onelineskip}
\noindent
\textbf{Palavras-chave}: robô móvel, raspberry pi, arduino, visão computacional, opencv, yolo.
\end{resumo}

% --- resumo em inglês ---
% \begin{resumo}[Abstract]
%     \begin{otherlanguage*}{english}
%         Abstract here.
        
%         \vspace{\onelineskip}
%         \noindent
%         \textbf{Keywords}: latex. abntex. publication de textes.
%     \end{otherlanguage*}
% \end{resumo}


