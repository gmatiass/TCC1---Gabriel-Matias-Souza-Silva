%%%%%%%%%%%%%%%%%%%%%% EXEMPLOS %%%%%%%%%%%%%%%%%%%%%%%%

%%%%%%%%%%%% Exemplo de Equação %%%%%%%%%%%%
%\begin{equation}
%    \lambda_0 = 9,2669 \times 10^{-11} T_0 +1,5478 \times 10^{-6}.
%\end{equation}

%\begin{equation} \label{eq:Momento_completo}
%2 \left ( \frac{2 \pi n_{eff}}{\lambda_b} \right )=\frac{2\pi}{\Lambda},
%\end{equation}

%================================================================================
%%%%%%%%%%%% Exemplo de Imagem %%%%%%%%%%%%
%\begin{figure}[!hbtp]
%  \centering
%   \caption{Diagrama de funcionamento de um sistema completo de interrogação de uma FBG.}
%    \includegraphics[width = 0.8\textwidth]{Caps/Figs/diagrama_sistema.eps}
%   \label{fig:sistema_original}
%    \fonte{Autoria própria}
%\end{figure}

%================================================================================
%%%%%%%%%%%% Exemplo de lista de itens enumerados e so com marcador %%%%%%%%%%%%

%\begin{enumerate}
%    \item Microcontrolador MSP430F5438A;
%    \item Conversor analógico-digital ADS1255;
%    \item Conversor digital-analógico DAC8568;
%    \item Isolador RS232/RS485 ADM2483;
%    \item Amplificador de Transimpedância.
%\end{enumerate}

%\begin{itemize}
%	\item MCLK: Clock mestre (\textit{Master clock}). Que pode usar qualquer umas das fontes como sinal. E é utilizado para a CPU e o sistema.
%	\item SMCLK: Subsistema mestre de clock (\textit{Subsystem master clock}). Também tem as cincos opções de fontes de entrada, porém é destinado e selecionado por software para módulos periféricos individuais.
%	\item ACLK: Clock auxiliar (\textit{Auxiliary clock}). Similar ao funcionamento do SMCLK.
%\end{itemize}

%================================================================================
%%%%%%%%%%%% exemplo para nomenclatura de sigla %%%%%%%%%%%%
%\nomenclature{USCI}{\textit{Universal Serial Comunication Interfaces}}

%================================================================================
%%%%%%%%%%%%% Exemplo de referencia de figura no texto %%%%%%%%%%%%
% demonstrado na figura~\ref{fig:pfrimer_2013_003}.

%================================================================================
%%%%%%%%%%%%% Exemplo de referencia %%%%%%%%%%%%
%@book{agrawal2014sistemas,
%  title={Sistemas de Comunica{\c{c}}{\~a}o por Fibra {\'O}ptica},
%  author={Agrawal, G.},
%  isbn={9788535264661},
%  
%  year={2014},
%  publisher={Elsevier Editora Ltda.}
%}

%@misc{demers2010fiber,
%  title={Fiber optic gyroscope},
%  author={Demers, Joseph R and Wong, Ka Kha and Logan Jr, Ronald T},
%  year={2010},
%  month=jun # "~29",
%  publisher={Google Patents},
%  note={US Patent 7,746,476}
%}

%@article{li2006fiber,
%  title={Fiber-optic temperature sensor based on interference of selective higher-order modes},
%  author={Li, Enbang and Wang, Xiaolin and Zhang, Chao},
%  journal={Applied physics letters},
%  volume={89},
%  number={9},
%  pages={091119},
%  year={2006},
%  publisher={AIP}
%}

%@book{agrawal2012fiber,
%  title={Fiber-optic communication systems},
%  author={Agrawal, Govind P},
%  volume={222},
%  year={2012},
%  publisher={John Wiley \& Sons}
%}